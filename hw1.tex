\documentclass[addpoints]{exam}

% Header and footer.
\pagestyle{headandfoot}
\runningheadrule
\runningfootrule
\runningheader{CS 113 Discrete Mathematics}{HW I: Sets}{Spring 2020}
\runningfooter{}{Page \thepage\ of \numpages}{}
\firstpageheader{}{}{}

% \qformat{{\large\bf \thequestion. \thequestiontitle}\hfill[\totalpoints\ points]}
\boxedpoints
\printanswers

\title{Homework I: Sets\\ CS 113 Discrete Mathematics\\ Habib University -- Spring 2020}
\author{Don't Grade Me}  % replace with your ID, e.g. oy02945
\date{}

\begin{document}
\maketitle

\begin{questions}


  \question[5]
  Write down $\mathcal{P}(X)$ if 
  $ X = \{ \emptyset, \{\alpha, \beta, \gamma \}, \gamma, \{\{ \alpha, \beta \} \} \}$.
  \begin{solution}
    % Write your solution here
  \end{solution}

  \question
  \begin{parts}
    \part[5] 
    Assume that RO has asked for your help to generate a set that contains all the possible pairs of DSSE faculty and DSSE courses at Habib University. Describe the sets and set operations that you can use to provide RO the desired set.
    \begin{solution}
      % Write your solution here
    \end{solution}

    \part[5] Imagine that the the operation above is extended to include an additional set that contains all the time slots when a course can be scheduled. Explain the outcome of the obtained set.
    \begin{solution}
      % Write your solution here
    \end{solution}
  \end{parts}
  
  \question
  The \textit{symmetric difference} of two sets $A$ and $B$ is defined as $A\oplus B = (A-B) \cup (B-A)$. It is also known as the \textit{disjunctive union} as it contains all those elements which are in either of those sets, but not in their intersection. 
  \begin{parts}
    \part[5] Prove that $A\oplus B = (A \cup B)-(A \cap B).$
    \begin{solution}
      % Write your solution here
    \end{solution}

    \part[10] For three sets $A, B,$ and $C$ we definine the symmetric difference as $A\oplus B\oplus C = (A\oplus B)\oplus C $, meaning using the definition twice. Draw the Venn diagram of this set and express it as in part (a).
    \begin{solution}
      % Write your solution here
    \end{solution}
  \end{parts}

  \question
  Let $A$ be the set of all numbers that are divisible by 6 and $B$ the set of all numbers that are divisible by $10$.

  \begin{parts}
    \part[5] Write the sets $A$ and $B$ in proper set notation and describe $A \cap B$ as simply as possible.
    
    \begin{solution}
      % Write your solution here
    \end{solution}
    
    \part[5] What is $A \oplus B$? Describe it using set notation and prove that it is indeed the symmetric difference of $A$ and $B$.
    \begin{solution}
      % Write your solution here
    \end{solution}

    \part[5] List down the elements of $A$, $B$, and $A \oplus B$ if $U = \{x\in \mathcal{N} \mid x \leq 60 \}$.
    \begin{solution}
      % Write your solution here
    \end{solution}
  \end{parts}

  \question
  Show that $\overline{ A \cup \overline{B}} = \overline{A} \cap B$.
  \begin{parts}
    
    \part[5] by using set identities.
    \begin{solution}
      % Write your solution here
    \end{solution}
    
    \part[5] by proving that each set is a subset of the other.
    \begin{solution}
      % Write your solution here
    \end{solution}
  \end{parts}
\end{questions}

\end{document}
